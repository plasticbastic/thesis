\chapter{Research Design and Methodology}
\label{chap:methodology} This chapter delineates the methodology employed to achieve
the primary objective of this study: extending the explainability of path-based
knowledge graph recommender systems to explore "what if" scenarios. The methodology
is structured as follows:
\begin{enumerate}
	\item \textbf{Initial Recommendation}: The process commences with the system generating
		a product recommendation for a user. In a path-based recommendation system, the
		inherent explanation typically comprises a sequence of entities and relationships
		that link the user to the recommended product.

	\item \textbf{Counterfactual Analysis}:
		\begin{itemize}
			\item \textbf{Extraction of Relevant Information}: Initially, the analysis
				extracts pertinent information related to the entities along the
				recommendation path. This includes the specific attributes that influenced
				the selection of the final product, as well as other potentially
				relevant attributes associated with the recommended product.

			\item \textbf{Scenario Construction}: Utilizing the extracted information,
				a collection of hypothetical scenarios is constructed. These scenarios are
				crafted to test various alterations in attributes and their impact on the
				recommendation outcome.
		\end{itemize}

	\item \textbf{Recommendation System Utilization}: For the recommendation engine,
		we employ CAFE (Coarse-to-Fine Neural Symbolic Reasoning for Explainable Recommendation).
		This system is particularly suitable for our purposes due to its path-based nature
		and its capability to evaluate the plausibility of different paths by assigning
		probability scores to the steps that connect users to products.
\end{enumerate}

This methodology both facilitates a deeper understanding of the decision-making processes
inherent in the recommender system and also allows us to simulate and evaluate how
changes in product attributes or user-product relationships might alter the system's
recommendations.

\section{Recommender System Details}
CAFE (Coarse-to-Fine Neural Symbolic Reasoning for Explainable Recommendation) is
used as the foundational framework for our recommender system. This section
provides an overview of its implementation and core functionalities.
\subsection{Data and Implementation}
The CAFE model is implemented using the Amazon review dataset, the beauty
category, which includes comprehensive user and product interactions. It
leverages predefined embeddings train in the model developed by \textcite{ai_learning_2018} described
in the previous section, as input to their symbolic network.
\subsection{Knowledge Graph Composition}
The knowledge graph at the heart of this recommendation system is intricately structured,
comprising several types of entities and relationships:
\begin{itemize}
	\item \textbf{Users} are linked to the words they have used and the products
		they have purchased.

	\item \textbf{Products} are associated with descriptive words, their brand,
		category, and other related products. Relationships with related products
		include those that have been 'bought together', 'also viewed', and 'also bought'.

	\item \textbf{Brands and Categories} form additional nodes, creating multiple pathways
		that connect different aspects of the data.

	\item \textbf{Related categories} mentioned above.
\end{itemize}

\subsection{Path-Based Recommendation Mechanics}
The system operates on predefined metapaths that represent meaningful
relationships leading a user to a product. These metapaths are substantial for understanding
the logic behind the recommendations. Todo: That paths . The recommender system assigns
a probability score to each step along the path, determining the strength of the
connection between the user and the potential product recommendations. It then selects
the top 10 paths with the highest cumulative scores, and the products associated
with these paths are recommended to the user. This scoring and selection process
ensures that the recommendations are both relevant and tailored to the user's preferences
and behavior patterns. This structured approach allows the CAFE system to
recommend products effectively but also provide insights into the reasons behind
each recommendation, providing explainability to the system.

\section{Counterfactual Analysis}
subsection{Community Detection and Graph Analysis Node Filtering}The counterfactual
analysis begins with the detection of communities within the knowledge graph of interactions.
Communities are identified using Louvain Method. This helps to cluster entities that
share significant similarities and interactions. The Louvain Method is a an
efficient algorithm designed for detecting communities in large-scale networks
by optimizing modularity, a measure that quantifies the density of links within communities
relative to those between them introduced in \parencite{blondel_fast_2008}. The algorithm operates
in two iterative phases: initially, it optimizes modularity locally, evaluating
potential gains by moving individual nodes into different communities. Nodes are
shifted to the community that maximizes this gain, and the process is repeated
until no further improvement is possible, achieving a local maximum of modularity.
In the second phase, the method aggregates these identified communities into new
nodes of a reduced network, and the process is reapplied. This hierarchical approach
allows the algorithm to uncover community structures at multiple levels effectively.
Notable for its speed, the Louvain Method can handle networks with up to millions
of nodes efficiently, making it well-suited for modern datasets of substantial
size. To refine the analysis further, we calculate the degree centrality for
each type of pair within the graph. Nodes that do not provide significant insight
are filtered out based on their z-score; specifically, nodes with a z-score exceeding
[specific threshold] are removed.

\subsection{Attribute Selection}
Following the predictions provided by the recommender system, a threshold is set
for the minimum score required for a product path to be recommended to a user, which
is the path score of the last product recommended in the top 10 recommended products.
For the analysis of a recommended path, we retrieve the first-level attributes and
their related products. This forms the initial layer of attribute selection, predicated
on the hypothesis that first-level connected items possess more relevant attributes.
These selected attributes are then evaluated to determine whether they fall
within the community of the recommended product. If they do, and their z-score
is within an acceptable range, they are considered for further counterfactual analysis.

\subsection{Performing Counterfactual Analysis}
For each attribute, an appropriate metapath is selected based on the type of the
attribute. Using the recommender engine, we calculate the score for a user-product
combination, which incorporates all the products previously purchased by the user
and the counterfactual attribute in question. This approach is grounded in the
assumption that the system discerns the user's preferences through their
purchase history. If the recalculated score for a product, when considering a
counterfactual attribute, exceeds the set minimum score, the attribute is
considered a positive influence for a product similar to the recommended one. This
isolated attribute analysis not only aids in understanding the influence of specific
attributes on product recommendations but also provides marketing insights. Such
insights can further be used to enhance the diversity and precision of the
recommender system.

\section{Case Study}