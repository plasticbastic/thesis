\chapter{Literature Review}
\label{chap:literature_review}

\section{Explainable AI In Knowledge Graph Recommender Systesm}

Knowledge Graphs (KGs) are pivotal in enhancing the explainability and accuracy of
recommender systems. These structured, relational frameworks capture complex interactions
among users, items, and their attributes, allowing for more nuanced
recommendations coupled with clear, logical explanations. This literature review
synthesizes recent advancements in explainable artificial intelligence (XAI)
that utilize KGs to illustrate how these technologies not only refine
recommendation quality but also enhance user trust and understanding through
transparency.

\subsection{Path-based Modeling}

Path-based modeling has emerged as a fundamental innovation in the utilization of
KGs for recommender systems. Techniques such as the Knowledge-aware Path
Recurrent Network (KPRN) and Path Language Modeling Recommendation (PLM-Rec)
illustrate this trend's dynamic nature. KPRN leverages LSTM networks to interpret
paths of entities and relationships, emphasizing those connections that are most
influential in understanding user preferences. This method enriches the recommendation
process by providing a temporal and semantic depth that traditional models lack,
allowing for a better prediction of user behavior based on past interactions \parencite{wang_explainable_2019}. 
On the other hand, PLM-Rec employs a novel approach by
integrating natural language processing techniques to extend KG paths. This model
treats paths as sentences, using a language model to dynamically predict and
extend these paths within the KG. Such extensions help the system explore new,
potentially uncharted areas of the KG, thereby enhancing the system’s ability to
recommend items that were previously unreachable. This approach addresses the
inherent limitations of static KG structures and improves the system's recall
capabilities, making it particularly valuable for discovering long-tail items \parencite{geng_path_2022}. 
Together, these path-based methods signify a shift towards more dynamic
and exploratory use of KGs, expanding both the depth and breadth of what
recommender systems can achieve.

\subsection{Integration of User Profiles and Behavior}

The integration of user behavioral data into KGs has significantly refined the personalization
capabilities of recommender systems. The "Cafe" model by \textcite{xian_cafe_2020} represents
a sophisticated application of this concept, employing a coarse-to-fine strategy
where initially broad user profiles help to narrow down and guide the path-finding
algorithms in KGs. These profiles are crafted from historical data and are instrumental
in focusing the recommendation process on paths most relevant to individual users,
thus enhancing both the relevance and personalization of the recommendations. This
method mirrors strategies used in other models that combine knowledge-base embeddings
(KBE) with collaborative filtering. By embedding user behaviors and item
characteristics into a unified representation, these models achieve a granular understanding
of user-item relationships. This integration allows for a tailored
recommendation experience, where the system's outputs are closely aligned with individual
preferences and behaviors, as demonstrated in the work by \textcite{ai_learning_2018}.

\subsection{Explainability through Symbolic and Logical Reasoning}

The demand for explainability in AI has driven the adoption of models that
incorporate transparent, logical reasoning processes. Monotonic GNNs (MGNNs), introduced
by Tena et al. (2022), exemplify this trend by ensuring that every transformation
within the network adheres to a set of logical rules, akin to traditional rule-based
systems. This adherence guarantees that the network's operations are interpretable
and justifiable, enhancing user trust by providing comprehensible explanations
for the recommendations made. Similarly, the Policy-Guided Path Reasoning (PGPR)
model uses reinforcement learning to navigate through the KG, selecting paths that
not only lead to relevant recommendations but are also interpretable. This model
provides explicit paths that detail the reasoning behind each recommendation, fulfilling
the dual requirements of accuracy and transparency in the recommendation process
\parencite{xian_reinforcement_2019}.

The convergence of these methodologies highlights a crucial trend towards enhancing
both the predictive accuracy and the interpretability of KG-based systems.
Through the integration of dynamic path exploration, personalized user profile analysis,
and logical reasoning, these approaches offer a more profound understanding of the
intricacies involved in making recommendations. They collectively emphasize a
shift towards recommender systems that are not only effective in their
predictions but also provide transparent and understandable explanations, aligning
with the growing user demand for transparency and accountability in AI systems.

\section{Counterfactual Methods in Different Types of Recommender Systems}

Counterfactual reasoning in recommender systems has emerged as a pivotal
technique within the domain of explainable artificial intelligence (XAI), enhancing
both the transparency and fairness of recommendations. By modeling alternative
scenarios where specific variables are modified, this approach provides insights
into the potential impacts of different data configurations, helping to
elucidate the inner workings and dependencies within these systems.

The introduction of the KGCR model \parencite{wei_causal_2023} marks a significant advancement
in embedding causal inference within graph-based recommender systems. Utilizing
Graph Convolutional Networks, this model enriches user, item, and attribute embeddings,
which allow for a more nuanced understanding of user preferences. By constructing
a causal graph and applying do-calculus interventions, the KGCR model
effectively mitigates biases introduced by previous user interactions, offering a
refined approach to understanding how bias influences recommendation outcomes.

In a similar vein, \textcite{tran_counterfactual_2021} developed the ACCENT framework, which
facilitates the generation of actionable counterfactual explanations in neural
recommender systems. This framework leverages extended influence functions to explore
how changes in user-item interactions could affect recommendation outputs,
significantly enhancing computational efficiency through Fast Influence Analysis.
This methodology underscores the minimal adjustments in user behavior that could
lead to different recommendations, thereby aiding in the creation of more
transparent recommendation mechanisms.

Addressing selection bias, \parencite{liu_practical_2022} implemented counterfactual policy
learning to recalibrate recommendation fairness and effectiveness. Their approach
utilizes Inverse Propensity Scoring to weigh observed interactions, allowing the
system to simulate outcomes under different recommendation policies. By integrating
these counterfactual outcomes into the learning process, the model achieves an
improved balance, enhancing both the performance and equity of recommendations across
various user groups and item categories.

The Prince method \parencite{ghazimatin_prince_2020}, emphasizes the
importance of trust and understanding in recommendation systems through
counterfactual reasoning within heterogeneous information networks. By identifying
key user actions and employing Personalized PageRank, Prince efficiently
predicts the impact of these actions on recommendation outcomes. This approach not
only avoids exhaustive computations but also outperforms traditional heuristic methods
in providing understandable and trust-enhancing explanations.

\textcite{yang_top-n_2021} utilize causal inference through Structural Equation Models (SEMs)
to address data sparsity in recommender systems. By generating counterfactual training
samples, they enrich the dataset with diverse user responses that are otherwise
not observed but plausible. This approach not only enhances the performance of the
recommender systems but also strengthens their capacity to handle scenarios marked
by data imbalance.

Finally, the Counterfactual Explainable Recommendation (CountER) model \parencite{tan_counterfactual_2021} focuses on identifying minimal attribute changes that could
reverse a recommendation decision. Through a structured optimization process, CountER
iteratively adjusts item attributes to discover the least extensive yet impactful
changes required for altering outcomes. This model utilizes novel metrics to
evaluate the necessity and sufficiency of these changes, demonstrating enhanced precision
in providing actionable insights into recommendation decisions.

In conclusion, counterfactual reasoning offers a robust framework for enhancing
the explainability and fairness of recommender systems by providing a deeper
understanding of the implications of various data interactions and policies. These
innovative approaches not only clarify the decision-making processes but also foster
more equitable and user-centric recommendation practices.

\section{Application of Counterfactuals for Fairness and Bias Mitigation}

Counterfactual reasoning plays a pivotal role in the domain of explainable
artificial intelligence (XAI), especially for mitigating biases in automated decision-making
systems. This method involves hypothesizing alternative scenarios where key
variables are altered, allowing for the exploration of how such changes impact outcomes.
This not only uncovers hidden biases but also ensures fairness in AI operations.
Broadly applied in various AI frameworks, from graph neural networks to
recommender systems, counterfactual reasoning enhances transparency and equity in
AI outcomes, establishing it as an essential tool for ethical AI development.

\subsection{Mitigating Bias Across Different AI Frameworks}

The use of counterfactual reasoning in graph-based models like those studied \textcite{guo_towards_2023} demonstrates a rigorous approach to maintaining consistency in
model predictions across varying sensitive attributes. By implementing Graph Variational
Autoencoders (GraphVAE), they not only perturb attributes but also train the
network to minimize discrepancies in outputs between the original and
counterfactual nodes. This methodology effectively addresses biases at a fundamental
level, ensuring the fairness of the model's outcomes. \textcite{medda_gnnuers_2024} extend
this approach within graph neural network-based recommender systems. Their innovative
use of counterfactual reasoning to adjust user-item interactions on a bipartite graph
includes strategically adding or removing connections, which serves to simulate
various scenarios where demographic disparities can be analyzed and mitigated, ensuring
a more equitable distribution of utility among users. The field of recommender
systems frequently grapples with biases such as popularity and exposure, which can
distort user preferences. \textcite{wei_model-agnostic_2021} dissect these issues through the Model-Agnostic
Counterfactual Reasoning (MACR) framework, which explicitly separates the
influence of item popularity from actual user preferences. By adjusting input data
to simulate a scenario where item popularity is neutralized, MACR provides a
recalibrated basis for recommendation, aligning more closely with unbiased user preferences.
Meanwhile, \textcite{xu_adversarial_2020} focus on exposure bias by employing a counterfactual
approach that involves a minimax adversarial model. This model simulates worst-case
scenarios to test the resilience of the recommendation system, ensuring that it
can withstand and adapt to a range of user exposure conditions, thus promoting a
more fair and balanced recommendation landscape.

\subsection{Enriching Data and Ensuring Equitable Outcomes}

Addressing data sparsity and imbalance, \textcite{yang_top-n_2021} utilize causal
inference via Structural Equation Models (SEMs) to generate counterfactual
scenarios that enrich training datasets. This not only addresses the immediate issue
of insufficient data but also simulates a broader spectrum of user interactions,
which helps in developing a more robust and responsive recommender system. On a more
focused level, \textcite{chiappa_path-specific_2019} pioneers the use of Path-Specific Counterfactual
Fairness (PSCF) within decision-making processes. This approach manipulates causal
pathways, particularly those that might be influenced by sensitive attributes
such as race or gender, to ensure that resulting decisions are free from the undue
influence of these attributes, thus promoting fairness in critical decision-making
contexts.

\subsection{Leveraging Knowledge Graphs for Fair Recommendations}

Expanding the utility of knowledge graphs, \textcite{balloccu_post_2022} integrate counterfactual
reasoning within the Policy-Guided Path Reasoning (PGPR) model to optimize recommendation
systems. By re-ranking items and explanations based on various fairness-oriented
criteria, such as recency, popularity, and diversity, PGPR enhances the quality
and equity of recommendations. This approach not only improves the relevance of the
recommendations but also significantly increases user trust and satisfaction by ensuring
that recommendations cater equitably to diverse user groups.