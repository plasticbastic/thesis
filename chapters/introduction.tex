\chapter{Introduction}\label{chap:introduction}

\section{Background and Motivation}
The rapid growth of the Internet as a center for digital and commercial
interactions has highlighted the essential role of recommender systems in today's
technology landscape. These systems, which are increasingly vital to companies
from media giants like Netflix to e-commerce leaders such as Amazon, utilize user
feedback to create personalized experiences. This feedback varies from explicit
ratings to implicit behavioral data, such as browsing and purchasing histories,
and serves as a critical foundation for these systems. By analyzing past interactions
between users and products, recommender systems can predict future preferences
and use advanced algorithms to detect patterns in user behavior. These capabilities
not only improve user satisfaction by aligning recommendations with individual tastes
but also enhance business strategies by promoting customer engagement and retention.
As user data grows more complex, these systems evolve by integrating sophisticated
models that consider content attributes, user specifications, and contextual information,
% thus increasing the accuracy and relevance of the recommendations offered
% \parencite{aggarwal_recommender_2016}. Despite their effectiveness in predicting user
% preferences and enhancing service personalization, traditional recommender systems
% often operate as opaque entities. This opacity arises from their complex
% computational mechanisms that typically use deep learning methods, known for their
'black-box' nature, where the internal reasoning processes remain hidden from
both users and developers. This lack of transparency poses significant challenges,
especially from a business perspective, where understanding and trust in these systems
are crucial. For businesses, the drive to make recommender systems explainable is
motivated by several key factors. Transparency builds trust; when users understand
how recommendations are generated, they are more likely to trust and accept
these suggestions, thereby increasing user satisfaction and loyalty. This is
particularly important in sectors like e-commerce or content streaming, where the
accuracy of recommendations directly affects user engagement and retention.
Furthermore, explainability aids in complying with regulatory standards, such as
the European Union's General Data Protection Regulation (GDPR), which requires that
users must understand the decisions made by automated systems that affect them.
This regulation is critical for businesses to ensure that their AI systems can
provide clear explanations for their outputs. Additionally, from a business
innovation standpoint, explainability enables the refinement and optimization of
recommender systems. By understanding the decision-making processes, developers can
identify and address biases or errors, leading to more accurate and fair
recommendations. This not only improves the system's performance but also increases
its fairness and accountability, aligning with ethical AI practices. The development
of explainable AI (XAI) seeks to clarify AI operations, allowing both users and
developers to understand, trust, and effectively manage AI outputs. Recent trends
toward this field highlight the growing recognition of the need for AI systems,
particularly recommender systems, to be both effective and understandable. As
such, the business case for explainable recommender systems is strong, promising
not only to enhance user experience and comply with regulations but also to
foster technological advancement and ethical AI practices \parencite{vultureanu-albisi_recommender_2021}
Explainable recommender systems can be understood from different angles. One way
for their classification is their approach to explainability.

\subsubsection{Types of Explainable Recommender Systems}

\begin{itemize}
	\item \textbf{Intrinsic Explainability:} This approach involves developing
		interpretable models where the decision-making process is transparent. As a result,
		it's easier to provide natural explanations for the decisions made by the
		model.

	\item \textbf{Model-Agnostic Approach:} Also known as the post-hoc explanation
		approach, this allows the decision mechanism to remain a black box. Instead
		of making the decision process itself clear, it focuses on developing a separate
		explanation model that generates explanations after a decision has been made.
\end{itemize}

The underlying philosophy of these approaches aligns closely with theories in
human cognitive psychology. Sometimes, we make decisions through careful, rational
reasoning and can clearly explain why we made those decisions. At other times, we
make decisions first and then rationalize them afterward to support or justify our
actions. There are multiple approaches to achieving intrinsic
explainability in recommender systems. These include factorization-based, topic modeling,
graph-based, specialized deep learning methods, knowledge-based, and rule mining
approaches. Knowledge graphs, in particular, play a critical role in advancing
the field of Explainable Artificial Intelligence (XAI) by providing a structured
and semantically rich framework that enhances the interpretability of complex AI
models. As \textcite{tiddi_knowledge_2022} describe, knowledge graphs are typically
structured as directed, edge-labeled graphs that describe entities and their relationships,
often organized within an ontological schema covering a variety of topics. This organization
reflects human cognitive processes of understanding and reasoning, making it a
transparent layer in AI systems where the decisions of otherwise

opaque models can be traced and understood through clear, logical pathways. The primary
value of knowledge graphs in explainability stems from their ability to connect AI
outputs with comprehensible and verifiable pieces of information, offering
deeper insights into the reasoning behind AI decisions. Therefore, knowledge
graphs not only enhance the trustworthiness and clarity of AI systems but also
play a significant role in bridging the gap between human understanding and
machine reasoning, crucial for the broader adoption and ethical deployment of AI
% technologies \parencite{tiddi_knowledge_2022}. Within the realm of knowledge graph-based
% recommender systems, two primary methodologies are prominent: embedding-based methods
and path-based methods. Embedding-based approaches capitalize on the rich
semantic information available within knowledge graphs to improve the
representation of users and items. These methods may focus on creating item-specific
graphs that enrich item representations with attributes from the knowledge graph
or develop user-item graphs that include both entities and utilize the defined relationships
and attributes to predict user preferences. Conversely, path-based methods leverage
the connectivity patterns within the graph, utilizing paths to define and
measure the similarity and interaction context between users and items. These methods
often employ predefined meta-paths to capture semantic similarities and refine representations,
thereby enhancing the recommendation process \parencite{guo_survey_2020}. While these
explanations significantly improve transparency, they often do not fully meet the
human cognitive need for a deeper understanding. Users frequently question how
recommendations might change if product attributes were altered, such as brand changes
or descriptors like "animal cruelty-free" or "plant-based." This study aims to
extend the level of explainability of path based kg recommendation through
counterfactual analysis. Counterfactuals represent hypothetical alternatives to actual
events, offering insights into what could have happened under different
circumstances. These constructs are crucial in various domains of artificial intelligence
(AI), especially in enhancing the interpretability and trustworthiness of AI
systems—a fundamental aspect of explainable AI (XAI). Counterfactuals serve as a
critical tool in XAI by providing "what if" analyses that elucidate the decision-making
processes of AI models, showing how alternative inputs might alter an AI
% systems output and making these systems more transparent and understandable. The
application of counterfactuals in XAI are present in both intrinsic and post hoc
approaches. Intrinsic methods integrate counterfactual reasoning as part of the
AI model's fundamental design, directly influencing its operational framework.
Conversely, post hoc techniques generate counterfactual explanations after the model
has made a decision. By simulating different scenarios, counterfactuals provide insights
into the specific conditions under which the outcomes of AI decisions might change,
thereby highlighting the causal relationships within the model's reasoning. These
"what if" scenarios do more than alter outcomes; they enrich the AI's decision-making
process, allowing users to explore and understand the model's behavior under various
hypothetical conditions. This aspect of counterfactuals in XAI not only aids in
debugging and improving AI systems but also enhances user trust, making AI decisions
more relatable and understandable, thereby fostering a deeper human-AI rapport \textcite{byrne_counterfactuals_2019}.

\section{Research Problem and Objective}

The central challenge addressed by this thesis is the limited explainability in
path-based knowledge graph recommender systems. These systems are adept at stating
the reasons for selecting a particular product but often lack the capability to delve
into hypothetical "what-if" scenarios, such as the impact of different
attributes on recommendations. Specifically, the systems do not answer questions
like, "What if the recommended product had different attributes? Which attributes
would still make it a plausible choice? Through what other attributes the
recommended product would be still plausible" This thesis addresses this gap by applying
counterfactual analysis to these systems, aiming to explore how hypothetical
changes to product attributes and interactions could alter the generated
recommendations, thereby enhancing the system's explainability. The primary
objectives of this research are twofold:

\begin{enumerate}
	\item \textbf{Development of an Analytical Model:} To create an analytical
		model that evaluates how changes in the attributes of recommended items
		affect the recommendation outcomes. This model is designed to identify attributes
		that would continue to uphold the recommendation's validity, providing
		deeper insights into the decision-making process of the recommender system.

	\item \textbf{Marketing and User Experience Insights:} To leverage the
		insights gained from the analytical model to offer marketing intelligence, foster
		diversity in recommendations, and formulate user profiles that influence
		these recommendations. This objective aims to enhance the way businesses understand
		and cater to diverse consumer preferences, thus improving user satisfaction
		and engagement.
\end{enumerate}

By meeting these objectives, the research will not only advance the theoretical
understanding of explainable AI in recommender systems but also equip businesses
with practical tools to refine their marketing strategies and enhance
recommendation processes.

\section{Methodological Approach}

The methodological approach developed in this thesis represents a novel strategy
for enhancing the explainability of knowledge graph-based recommender systems through
counterfactual analysis. This approach employs techniques in data processing and
graph analytics to simulate potential scenarios under various attribute
conditions. By manipulating attributes and observing the resultant changes in recommendations,
this method offers a detailed view into the decision-making process of the
recommender system.

\section*{Key Components}

The methodology consists of several essential components:

\begin{enumerate}
	\item \textbf{Attribute Extraction:} This process involves identifying and
		extracting relevant attributes and interactions of recommended products from
		the knowledge graph. This step is crucial for understanding the specific factors
		that might influence the recommendations.

	\item \textbf{Community Identification:} Utilizing community detection
		algorithms helps manage the complexity and size of the data by grouping
		similar products and attributes. This organization enhances the relevance and
		efficiency of the counterfactual analysis by ensuring that the simulations are
		grounded in realistic and comparable scenarios.

	\item \textbf{Recommendation Scoring:} This component evaluates each
		hypothetical scenario by calculating the recommendation scores for altered
		paths. It allows for a comparison with actual recommendation outcomes to determine
		if a product with changed attributes would still be recommended. This
		scoring process is integral in assessing the robustness and flexibility of
		the recommender system under varied conditions.
\end{enumerate}

\section{Expected Contributions and Outcomes}

\textbf{Contributions to the Field:} This research contributes to the fields of
recommender systems and explainable AI by introducing a counterfactual framework
that incorporates counterfactual reasoning into the recommendation process. This
adds a significant layer of depth to the analytical capabilities of existing systems
and enhances the explainability of complex recommender systems, bridging the gap
between advanced algorithmic decisions and user comprehensibility.


\textbf{Anticipated Findings:} The expected findings of this research are likely to show that
certain attributes and their changes have significant impacts on recommendations,
highlighting the importance of these attributes in the recommendation logic. By providing
clearer explanations for recommendations, user satisfaction and trust in the
system are expected to improve. Additionally, insights from counterfactual
analyses can inform more targeted marketing strategies and product development
initiatives, leading to better customer targeting and enhanced product offerings.
These contributions and findings are anticipated to foster a deeper
understanding of recommendation systems, promoting more informed and transparent
use in various applications.

\section{Contribution}

This research advances the fields of recommender systems and explainable AI by
integrating counterfactual reasoning into the path-based knowledge graph
recommendation process. This integration introduces a depth to the analytical capabilities
of existing systems and enhances the explainability of complex recommender systems,
effectively bridging the gap It aligns algorithmic decision-making more closely
with stakeholder needs and expectations, thereby increasing the practical utility
and transparency of these systems.

\section{Structure of the Thesis}

% \section*{Thesis Organization}

This thesis is organized into several chapters, each serving a specific purpose
to comprehensively cover the research undertaken. Here's a brief outline of each
chapter and what it entails:

\textbf{Chapter 1: Introduction}

This opening chapter sets the stage for the research by introducing the topic,
stating the research problem, and outlining the objectives. It provides a
thorough background on the importance of explainability in knowledge graph-based
recommender systems and introduces the concept of counterfactual analysis as a
novel approach to address these challenges.

\textbf{Chapter 2: Literature Review}

The literature review chapter delves into existing studies and theories relevant
to knowledge graph recommender systems and explainable AI. It discusses previous
approaches to improve transparency in AI-driven recommendations, evaluates their
limitations, and highlights the need for innovative solutions like counterfactual
analysis. This review establishes the theoretical foundation for the research
and identifies gaps that this thesis aims to fill.

\textbf{Chapter 3: Methodology}

In this chapter, the research methodology used in the thesis is detailed. It
describes the development of the counterfactual framework, including the processes
of attribute extraction, community identification, and the computation of recommendation
scores. The chapter elaborates on the tools and algorithms employed, the data collection
process, and the techniques used to simulate and analyze various counterfactual
scenarios.

\textbf{Chapter 4: Results}

The results chapter presents the findings from the application of the
counterfactual analysis framework. It includes detailed discussions on how different
attributes and their hypothetical modifications impact the recommendation outcomes.
The results are supported by quantitative data and visual representations to
illustrate the effects of attribute changes on the recommendations.

\textbf{Chapter 5: Conclusion and Recommendations}

The concluding chapter synthesizes the findings, discussing the implications for
both theory and practice. It evaluates the success of the counterfactual
framework in enhancing the explainability of recommender systems and suggests
areas for future research. Recommendations for practitioners in the field of recommender
systems are also provided, focusing on how they can implement similar
methodologies to improve the transparency and effectiveness of their systems.